\addcontentsline{toc}{part}{Annexes}
\part*{Annexes}
 
 \chapter*{Tableau d'inputs utilisateurs possibles}
%  \noindent
   \begin{tabu} to \linewidth{|X[-2.5,c,m]|X[c,m]|}
%    \toprule
     \tabucline-
     \textbf{Entrée utilisateur} & \textbf{Influence sur la simulation} \\  \tabucline- 
%    \midrule
     ECHAP               &  Arrêt de la simulation. \\ \tabucline-
     R                   &  Repositionne la caméra à son emplacement initial. \\ \tabucline-
     Z, Flèche avant     & Translation positive de la caméra sur l'axe des x. \\ \tabucline-
     S, Flèche arrière   & Translation négative de la caméra sur l'axe des x. \\ \tabucline-
     D, Flèche droite    & Translation positive de la caméra sur l'axe des y. \\ \tabucline-
     Q, Flèche gauche    & Translation négative de la caméra sur l'axe des y. \\ \tabucline-
     E                   & Translation positive de la caméra sur l'axe des z. \\ \tabucline-
     A                   & Translation négative de la caméra sur l'axe des z. \\ \tabucline-
     P                   & Rotation positive autour de l'axe des z. \\ \tabucline-
     O                   & Rotation négative autour de l'axe des z. \\ \tabucline-
%    \bottomrule
   \end{tabu}
   
   
 \chapter*{Sources}
  \begin{itemize}
   \item[$\rightarrow$] Logiciel utilisé pour créer le projet : QtCreator \newline \url{https://www.qt.io/download-open-source/}
   \item[$\rightarrow$] Logiciel utilisé pour écrire le rapport en LaTeX : ShareLatex \newline \url{www.sharelatex.com}
   \item[$\rightarrow$] Logiciel utilisé pour faire de la modélisation 3D : 3ds Max \newline \url{http://www.autodesk.fr/products/3ds-max/overview}
   \item[$\rightarrow$] Librairies utilisées pour le rendu graphique 3D : OpenGL \newline\url{https://www.opengl.org/}
   \item[$\rightarrow$] Librairie d'aide pour calculs mathématiques : glm \newline\url{http://glm.g-truc.net/0.9.6/index.html}
   \item[$\rightarrow$] Logiciel utilisé pour faire du versioning : Git \newline\url{https://git-scm.com/}
  \end{itemize}