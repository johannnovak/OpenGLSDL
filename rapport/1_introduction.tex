\chapter*{Introduction}
 Ce rapport a été réalisé dans le cadre de l'UV IN55 concernant un projet de rendu graphique 3D à réaliser à l'aide des librairies OpenGL sous l'environnement graphique QT. Parmi le sujets proposés, nous avons choisi le système de particule car c'est un thème très intéressant. Nous ne nous rendons pas forcément compte mais beaucoup de phénomènes physiques nous entourant peuvent être assimilés à un système de particule, plus ou moins complexe. Que ce soit le feu, la fumée, les nuages ou même la poussière environnante, tout peut-être représenté par un tel système. \\
\indent %TODO \\
\indent Nous avons décidé de partager ce rapport en trois parties. Tout d'abord nous aborderons le gros du sujet qui concerne l'architecture de notre moteur 3D. Ensuite dans une deuxième partie, nous présenterons deux scènes où nous décrirons pour chacun les différentes spécificités. Finalement dans une troisième partie, nous ferons un bilan du projet avec les performances du moteur, les difficultés rencontrées et les améliorations possibles.