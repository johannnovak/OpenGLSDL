\usepackage{lmodern}		% fonts et syntaxe
\usepackage[T1]{fontenc}	% direct accents en output
\usepackage[utf8]{inputenc}	% direct accents en input
\usepackage{graphicx}   % include graphic
\usepackage{soul}		% souligner
\usepackage{url}		% utiliser des url
\usepackage{color}		% utilisation des couleurs
\usepackage{bookmark}		% extension des ref + bookmarks
\usepackage{array}		% pour tabular etc ...
\usepackage{amsmath}
\usepackage{mathrsfs}
\usepackage{geometry}		% layout du rapport
\usepackage{tabularx}		% pour tabularx
\usepackage{tabu}           % pour tabu
\usepackage{frenchle}		% nom des chapitres, etc... en FR
\usepackage[bottom]{footmisc} %for footnote mandatory bottom position
\usepackage{longtable}
\usepackage{float}


% Set des margins du document.
\geometry{bmargin = 2.9cm, tmargin = 2cm}

% Définition de couleurs perso.
\definecolor{gris}{rgb}{0.5,0.5,0.5}
\definecolor{darkgreen}{rgb}{0,0.39,0}
\definecolor{darkred}{rgb}{0.33,0.6,0.7}
\definecolor{violet}{rgb}{0.28,0.24,0.55}

% Permet de préciser la couleur du lien, supprimer la border et d'autoriser le partitionnage de l'URL
\hypersetup{linkcolor=black, urlcolor=blue, colorlinks= true, breaklinks=true, pdfborder= 0 0 0}

% Permet de faire un url break sur les '/' ou le '-'
\def\UrlBreaks{\do\/\do-}